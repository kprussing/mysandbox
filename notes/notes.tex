\documentclass[aps,prl,reprint]{revtex4-1}

% You'll need my preamble from my dot files.
\input{my_preamble}

\begin{document}
%--------1---------2---------3---------4---------5---------6---------7--
% Header matter
\title{My Working Notes}
\author{Keith Prussing}
\email{kprussing3@gatech.edu}
\affiliation{
    School of Physics \\
    Georgia Institute of Technology, Atlanta, GA, 30332
}
\date{\today}

\begin{abstract}
    This is a place where I can keep all of my working notes together
    for future use.  The use is primarily for myself, but anyone who
    wants to could use these notes.  I highly doubt that any new
    information is presented within; however, a novel perspective might
    very well be contained herein.
\end{abstract}
\maketitle

%--------1---------2---------3---------4---------5---------6---------7--
\section{Boundary Element Method}
In this section, I want to work out some details for the formulation of
the boundary element method (BEM).  This will be a little sporadic so
bear with me.

\subsection{Green's Equations}
In chapter 1 section 3 of~\cite{ang_beginners_2007}, he blows through a
couple of math steps that I want to work out.  We take \(u\) and \(v\)
to be solutions to the two dimensional Laplace equation
\begin{align*}
    \frac{\partial^2 u}{\partial x^2} + 
    \frac{\partial^2 u}{\partial y^2} &= 0, \\
    \frac{\partial^2 v}{\partial x^2} + 
    \frac{\partial^2 v}{\partial y^2} &= 0. 
\end{align*}
Multiplying the first by \(v\) and the second \(u\) and subtracting we
find
\begin{align*}
    v\frac{\partial^2 u}{\partial x^2} -
    u\frac{\partial^2 v}{\partial x^2} +
    v\frac{\partial^2 u}{\partial y^2} -
    u\frac{\partial^2 v}{\partial y^2} = 0.
\end{align*}
And here is the step he glossed over.  Simply add and subtract
\(\partial_x\,u\partial_xv\) to get
\begin{align*}
    v\frac{\partial^2 u}{\partial x^2} -
    u\frac{\partial^2 v}{\partial x^2} +
    \frac{\partial v}{\partial x}\frac{\partial u}{\partial x} -
    \frac{\partial v}{\partial x}\frac{\partial u}{\partial x} &= 0 \\
    \frac{\partial}{\partial x}\left( 
        v\frac{\partial u}{\partial x} -u\frac{\partial v}{\partial x}
    \right) &= 0
\end{align*}
using the product rule.  A similar process applies to the \(y\)
derivative.  This is just one way to generate Green's functions by the
looks of things.

\subsection{Laplace's Equation}
Okay, so I missed something along the way.  Let me back up and take a
look at deriving the equations in my own notation.  We wish to find a
solution to Laplace's equation in a region \(\Gamma\) bounded by the
surface \(S\).  We assume that either the value or the normal
derivative, but not both, of the field is specified on \(S\) at all
points.  We seek a solution to Laplace's equation that satisfies the
boundary conditions.  Our first task will be to formulate an integral
equation, and then we shall introduce an expansion set to convert to a
matrix formulation.

We begin our investigation in the 2D case.  In two dimensions, the
Laplace operator becomes
\begin{align*}
    \div\grad u(x,y) &= \frac{\partial^2 u}{\partial x^2}
    +\frac{\partial^2 u}{\partial y^2} \\
    \div\grad u(\rho,\theta) &= \frac{1}{\rho}
    \frac{\partial}{\partial\rho} 
    \left( \rho\frac{\partial u}{\partial\rho} \right)
    +\frac{1}{\rho^2}\frac{\partial^2 u}{\partial\theta^2}.
\end{align*}
The next part loosely follows the Wikipedia page on Green's identities,
but I know I've seen it in common texts in the literature.  I just don't
have any handy at the moment.  Now we take the divergence theorem
\begin{equation*}
    \int_\Gamma \div\svec{F}\,dV = \oint_S \unitvec{n}\cdot\svec{F}\,dS
\end{equation*}
with the vector field \(\svec{F} = u\div v -v\div u\) to find
\begin{equation*}
    \int_\Gamma u\div\grad v -v\div\grad u \, dV =
    \oint_S
    u\frac{\partial v}{\partial n} -v\frac{\partial u}{\partial n}
    \,dS.
\end{equation*}
This becomes the starting point for the three dimensional case.  In the
case of two dimensions, \(u\) and \(v\) have no functional dependence on
\(z\) and the Laplace operator is as defined above.  We may remove the
integration over \(z\) by simply inserting the Dirac delta function
\(\delta(z)\) and integrating.  Thus the equation of interest in two
dimensions is 
\begin{equation*}
    \int_\Gamma u\div\grad v -v\div\grad u \, dA =
    \oint_S
    u\frac{\partial v}{\partial n} -v\frac{\partial u}{\partial n}
    \,dS.
\end{equation*}
Boy that was pointless.  The equation looks the same.  I sure hope the
derivation of the Green's function is more interesting.

Next, we introduce the function \(g(\svec{x},\svec{x}')\) that is a
solution to the Poisson equation
\begin{equation*}
    \div\grad g(\svec{x},\svec{x}') = -\delta(\svec{x}-\svec{x}').
\end{equation*}
I know that Rother had a good description of this part for the Helmholtz
equation~\cite{rother_electromagnetic_2009}.  Plugging this into Green's
second identity above and taking \(u\) to satisfy Laplace's equation we
find
\begin{equation*}
    u(\svec{x}) = \oint_S \left[
        g(\svec{r},\svec{x})\grad[\svec{r}] u(\svec{r})
        -u(\svec{r})\grad[\svec{r}] g(\svec{r},\svec{x})
    \right] \cdot \unitvec{n} \, dS(\svec{r})
\end{equation*}
I should really use some equation labels.  In two dimensions, this will
simply reduce to the integral around the contour bounding the region
\(\Gamma\).  Note, that this is for a point \(\svec{x}\) that lies with
in the region \(\Gamma\).  The function \(u\) is not defined external to
this region.  But what happens, when \(\svec{x}\) lies on the surface
\(S\)?  Rodriguez and company state that there is a \(1/2\) but they
don't derive it~\cite{rodriguez_fluctuating-surface-current_2013}, Ang
derives it in 2D in what I can see~\cite{ang_beginners_2007}.  I guess I
should sketch out the proof.

Here we go!  Due to the singularity when \(\svec{x}\) lies on the
surface \(S\), we must interpret the surface integral as a principle
value~\cite{ang_beginners_2007}.  If we surround the point \(\svec{x}\)
by a circle of radius \(\eps\) and exclude this region from surface
integral, we know would get \(0\).  By letting \(\eps\) tend toward
zero, we would get the original form.  So, we need to evaluate an
integral of the form
\begin{equation*}
    \lim_{\eps\to0}
\end{equation*}
I need to look at some more formalisms.  In principle, if we exclude a
region of "radius" \(\eps\) from the surface integral we could then let
\(\eps\) tend toward \(0\).  This would then see a locally flat region
of \(S\) was smooth.  I'm not completely sure how to get that region to
essentially have a factor of \(-1/2\).  I should look into the
references in~\cite{rodriguez_fluctuating-surface-current_2013}.

Now, what is the Green's function in two dimensions?  First, we Fourier
transform the differential equation to the frequency domain
\begin{align*}
    \int_{\mathbb{R}_n} \div\grad g(\svec{r})
    e^{-i \svec{k}\cdot\svec{r}} d^nr &= -\int_{\mathbb{R}_n}
    \delta(\svec{r}) e^{-i\svec{k}\cdot\svec{r}} d^nr \\
    k^2 \tilde{g}(\svec{r}) &= -1 \\
    g(\svec{r}) &= \frac{-1}{(2\pi)^n} \int_{\mathbb{R}_n}
    \frac{e^{i\svec{k}\cdot\svec{r}}}{k^2} \, d^nk
\end{align*}
Now to derive the Green's function in various dimensions.

First, we start with \(n=3\).  This gives us
\begin{align*}
    g(\svec{r}) &= \frac{-1}{(2\pi)^3} \int_0^{2\pi}d\phi_k
    \int_0^\infty \int_{-1}^1 \frac{e^{ikru}}{k^2} \,du \,k^2 dk \\
    &= \frac{i}{(2\pi)^2r}
    \int_0^\infty \frac{e^{ikr} -e^{-ikr}}{k} \,dk \\
    &= \frac{i}{(2\pi)^2r} \int_{-\infty}^\infty \frac{e^{ikr}}{k} \,dk
\end{align*}
where the last step follow from separating at the subtraction,
substituting \(k' \to -k\), swapping the limits of integration,
substituting \(k \to k'\), and adding the integrals.  To perform the
integral, we project onto the complex plane and use the residue
theorem.  We subtract a small negative imaginary part to the denominator
and allow it to approach zero.  We close the contour in the upper half
plane where \(z\) has a large imaginary part so that the integrand goes
to zero.  Thus we find
\begin{align*}
    2\pi i \lim_{\eps\to0} e^{ikr} \Big\rvert_{k=0} &= \lim_{\eps\to0}
    \oint \frac{e^{izr}}{z-i\eps} \,dz \\
    2\pi i&= \int_{-\infty}^\infty \frac{e^{ikr}}{k} \,dk.
\end{align*}

\section{Radiative Transport}
In this section, we will discuss some musings about the general
properties of the radiative transport between grey bodies.  We assume
that all reflections are diffuse.  This implies that for any two bodies
at finite temperatures, the energy exchange between them is 
\begin{equation}
    H_{1,2} = \sigma A_1\mathcal{F}_{1,2} \left(T_1^4 -T_2^4\right).
\end{equation}
In the above, \(A_1\mathcal{F}_{1,2}\) is the total exchange factor 
described by Hottel and Sarofim [I need to figure out which book that
was].  I should really go take a look at those notes I made about all of
this a few years back.  They are on my desktop at work and the text
files are on Ursa.

\subsection{Previous Work Discussion}
The general means to compute the radiative transfer between two surfaces
has been studied extensively over the past half century.  The assumption
is made that all surfaces are diffuse grey body reflectors and that
Kirchoff's law holds at all wavelengths.  The energy exchange is
governed by the black-body exchange along with a factor representing the
relative exchange between the two bodies
\begin{equation}
    H_{ij} = \sigma(T_i^4 -T_j^4) A_i\mathcal{F}_{ij}.
\end{equation}
The challenge of the work is to determine how much energy leaving \(i\)
is absorbed by \(j\).  

If all surfaces are black, then the energy leaving \(i\) that is
incident on \(j\) is absorbed.  Because the energy travels in a straight
line, the relative about of are of \(i\) that can be ``seen'' by \(j\)
is our factor \(A_iF_{ij}\).  This factor \(F_{ij}\) is referred to in
the literature as the view factor~\cite{clark_algebraic_1974,
hottel_radiative_1967}, the angle factor~\cite{hottel_radiative_1967,
sparrow_new_1963}, and the configuration
factor~\cite{eichberger_calculation_1985}.  This factor is derived from
the differential amount of energy emitted by area element \(dA_i\) that
is incident on element \(dA_j\)~\cite{sparrow_new_1963}
\begin{equation}
    dF_{dA_i\to dA_j} = \frac{\cos\theta_i\cos\theta_j}{\pi r^2} dA_j.
\end{equation}
The total energy emitted by a finite surface incident on a finite
surface \(A_j\) is obtained by integrating over the surface
\begin{equation}
    F_{dA_i\to A_j} = \iint_{A_j} \frac{\cos\theta_i\cos\theta_j}{\pi
    r^2} \, dA_j,
\end{equation}
and the total energy exchanged between finite surfaces in is obtained by
a double surface integral~\cite{sparrow_new_1963}
\begin{equation}
    F_{A_i\to A_j} = \frac{A_i}\iint_{A_i}\iint_{A_j} 
    \frac{\cos\theta_i\cos\theta_j}{\pi r^2} \, dA_j dA_i
\end{equation}
When the bodies are grey, a
first approximation to account for emissivities other than unity is to
multiply by the emissivity of each surface
\(F_{ij}\Rightarrow\epsilon_i\epsilon_j
F_{ij}\)~\cite{eichberger_calculation_1985}.  While this is an
improvement, it does not account for multiple reflections.

To work around this problem, Hottel and Sarofim used a ``total view
factor'' \(\mathcal{F}_{ij}\) and the equivalent total exchange area
\(\overline{S_iS_j} = A_i\mathcal{F}_{ij} =
A_j\mathcal{F}_{ji}\)~\cite{hottel_radiative_1967}, and Gebhart used an
absorption coefficient \(B_{ij}\)~\cite{gebhart_surface_1961}.  While
numerically different, the base philosophy of both are the same.  Each
factor encompasses the entire energy emitted by \(i\) and absorbed by
\(j\).  This includes energy reflected by other surfaces in the
configuration, multiple reflections, or any other path one may consider.
Clark and Korybalski demonstrated the formal equivalence of both the
total view factor and the absorption coefficient and that both are
related to the geometric view factors~\cite{clark_algebraic_1974}.

One bottleneck is that the total view factors require the computation of
the geometric view factors first.  We must then solve \(N\) systems of
\(N\) equations.  We may see this by considering the fact that in order
to trace energy from one surface through all reflections in the
configuration, we must visit every surface.  The Southwell relaxation
method beautifully illustrates this~\cite{cohen_radiosity_1993}.  The
surface with the highest energy ``shoots'' its excess into the scene.
The energy is distributed across all surfaces according to the geometric
view factor.  The energy at each surface is adjusted according to the
absorptivity and the reflected energy is is queued up to be ``shot'' as
excess.  The procedure it repeated with each highest energy surface
until an equilibrium is reached and there is no energy remaining to
``shoot'' without breaking physical laws.  This is simply tracing the
energy through the scene in the same philosophy as the total view
factors and the absorption coefficients.

The second bottle neck is the computation of the view factors.  There
exist semi-analytic methods for axially symmetric non-occluded
cases~\cite{chung_simpler_1982, naraghi_radiation_1982,
chung_formulation_1984}, and there are catalogs of various non-occluded
cases~\cite{howell_catalog_2010}; however, these are for limited
configurations or provide limited precision.  Most numerical methods to
compute the view factors are over a quarter of a century old.  Recently,
Walton presented a new program using an adaptive meshing
algorithm~\cite{walton_calculation_2002}.  We know the view factor is
related to the solid angle~\cite{sparrow_new_1963}, so we could employ a
Monte Carlo approach similar to Whitcher~\cite{whitcher_monte_2012}, but
this would still require ray tracing to check for occlusions.

\subsection{Scale Invariance of View Factors}
Consider the exchange between two surfaces \(A_i\) and \(A_j\).  We know
the view factor is defined by
\begin{equation}
    A_iF_{ij} = \iint_{A_i}\iint_{A_j} 
    \frac{\cos\theta_i\cos\theta_j}{\pi r_{ij}^2}
    \,dA_jdA_i.
\end{equation}
The differential surface area is given by~\cite{tai_generalized_1997}
\begin{equation}
    dA = \left\lVert \frac{\partial \svec{r}}{\partial u} \times
    \frac{\partial\svec{r}}{\partial v} \right\rVert_{\svec{r}\in A}
    \,dudv.
\end{equation}
Now let us multiply all length scales by a common factor \(R\).  This
could simply be a unit conversion (eg. \SI{1}{\metre} =
\SI{1000}{\milli\metre}).  This means \(\svec{r}' = R\svec{r}\), \(x' =
Rx\), \(y' = Ry\), and \(z' = Rz\).  If \(u\) or \(v\) has dimensions of
length, we immediately see that the scaling factor will cancel in the
definition of the differential area 
\begin{equation}
    \frac{\partial\svec{r}'}{\partial u'}du' =
    R\frac{\partial\svec{r}}{\partial (Ru)} d(Ru) =
    R\frac{\partial\svec{r}}{\partial u} du.
\end{equation}
This makes the scaled differential area
\begin{align}
    dA' &= \left\lVert \frac{\partial \svec{r}'}{\partial u} \times
    \frac{\partial\svec{r}'}{\partial v} \right\rVert_{\svec{r}'\in A'}
    \,dudv 
    \nonumber \\
    &= R^2 \left\lVert \frac{\partial \svec{r}}{\partial u} \times
    \frac{\partial\svec{r}}{\partial v} \right\rVert_{\svec{r}'\in A'}
    \,dudv 
    \nonumber \\
    &= R^2 dA.
\end{align}
Substituting this into the definition of the view factor we find
\begin{equation}
    F_{ij} = \frac{1}{R^2A_i} \iint_{A_i'} \iint_{A_j'}
    \frac{\cos\theta_i\cos\theta_j}{\pi (Rr_{ij})^2} \,dA_j'dA_i',
\end{equation}
but \(r_{ij}' = Rr_{ij}\) and the two length scales in the surface
\(A_i\) are scaled so that \(A_i' = R^2A_i\) so 
\begin{equation}
    F_{ij} = \frac{1}{A_i'} \iint_{A_i'} \iint_{A_j'}
    \frac{\cos\theta_i\cos\theta_j}{\pi (r_{ij}')^2} \,dA_j'dA_i'.
\end{equation}
This means the geometric view factor is unchanged by a simple scaling of
all lengths.  In other words, the geometric view factor is the same
regardless of the system of units used to measure the areas.

It is intuitively obvious that the view factor should be scale
invariant.  I can increase the size of both objects by a common factor
which would increase the relative amount that each ``sees'' of the
other.  However, if I simultaneously increase the separation of all
objects, the relative amount that is ``seen'' is unchanged.  This is
analogous to the solid angle from which we can derive the view factor.
This means we can use vector sweep at a scaled version.

\subsection{Random Babbling About Stuff}
We know the total exchange factors for a system are dependent on the
geometric view factors \(F_{1,2}\) and the emissivity of the objects.
Once those are know, the total exchange factors are found by solving the
\(N\) systems of equations where \(N\) is the number of ``nodes'' in the
system.  I say ``nodes'' because Ken and I had a huge argument about
using the word ``nodes'' unqualified.  In the current context, I am
referring to a ``thermal node'' concept; however, that is not very
precise. 

Because the total exchange factor \(A_i\mathcal{F}_{i,j}\) captures all
of the multiple reflections, this will give the proper black-body
exchange.  I plotted the exchange normalized by the geometric view
factor \(F_{1,2}\) and saw that the oblate spheroid still displayed an
increase of \(\sim\)80\% while the prolate decrease \(\sim\)40\%.  These
numbers are just off the top of my head and are not precise.  If we
consider the fact that the oblate spheroid has a much larger geometric
view factor, we expect this.  But, because the local normal of an oblate
spheroid is closer to that of a plane, more of the energy can
participate in multiple reflections compared to a sphere or a prolate
spheroid.  This also explains why the sphere-sphere plot was still very
high as opposed to a little high.  Of course, it could simply be that
the near-field effects are huge like everyone says.  This means we do
need to get the total exchange factors and not just the geometric.  This
sounds like a job for vector sweep, but it probably doesn't want to work
with the small length scales (\textsc{F10.4} list directed
\textsc{fortran77} or something like that).  If we can establish that
the exchange factors are scale invariant, we may be able to work with
this.

We may just have to rewrite a program to compute these things.  There
are ``recent'' programs developed to do
this~\cite{walton_calculation_2002}, but I haven't gotten around to
grabbing the source code yet.  A brute force method would be to use a
Monte Carlo approach similar to Whitcher's solid angle
approach~\cite{whitcher_monte_2012}, but we know that is an \(O(N^3)\)
approach~\cite{walton_calculation_2002}.  We could use catalogs for the
well documented geometric view factors~\cite{howell_catalog_2010}, but
that will give us limited precision.  Chung and Naraghi have presented
method on computing the geometric view factor for axially symmetric
cases~\cite{chung_simpler_1982, naraghi_radiation_1982,
chung_formulation_1984}, but that is still the geometric factor.  In our
case, we are only considering two surfaces, so that might not be a
terrible thing once we can establish the emissivity of gold.  I'm still
not sure that we are getting the correct geometric view factor out of
the quick and dirty script.  It only matches the catalog to 2
significant figures~\cite{howell_catalog_2010}.  This leads us back to
scale invariance and vector sweep.

I also want to find the reference Ken was using.  I'm not sure if
Spencer was also using that reference, but I'll never know.  Spencer was
terrible about writing things down.  I know he used \textsc{pbrt}, but
that's all I know.  According to Wikipedia, the geometric view factor
between differential areas is
\begin{equation}
    F_{1\to2} = \frac{\cos\theta_1\cos\theta_2}{\pi S^2} dA_2.
\end{equation}
This is not very symmetric, but it probably is due to the reciprocity
relation \(A_1F_{1\to2} = A_2F_{2\to1}\).  The angles are those between
the normal vector for the surface patch and the vector connecting the
two center points \(\svec{S}\) and \(S\) is the distance between the two
\(S = \lVert\svec{S}\rVert\).  This is all well and good and I remember
this.  I need to dig around in how scaling the differential area
\(dA_2\) works.  Obviously, the separation is easily scaled \(S' = R S\)
where \(R\) is simply a desired ``scaling'' factor.  I use \(R\) because
I know I'll want to use the radius.

The differential patch is defined as [need source]
\begin{equation}
    dA \equiv \left\lVert \frac{\partial\svec{r}}{\partial u} \times
    \frac{\partial\svec{r}}{\partial v} \right \rVert.
\end{equation}
So, if we select \(u\) and \(v\) such that they are dimensionless, the
we \emph{should} be able to simply scale the position vector
\(\svec{r}\) by the same scaling factor \(\svec{r}' = R \svec{r}\).
I'll have to be cautious about this.  We might have to work with the
factor \(A_1F_{1\to2}\) to ensure the scaling works.  But that has units
of area so I'm not sure haw that would really work\ldots


%--------1---------2---------3---------4---------5---------6---------7--
% And the end matter.
\bibliography{references,notes}
\end{document}

