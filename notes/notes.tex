\documentclass[aps,prl,reprint]{revtex4-1}

% You'll need my preamble from my dot files.
\input{my_preamble}

\begin{document}
%--------1---------2---------3---------4---------5---------6---------7--
% Header matter
\title{My Working Notes}
\author{Keith Prussing}
\email{kprussing3@gatech.edu}
\affiliation{
    School of Physics \\
    Georgia Institute of Technology, Atlanta, GA, 30332
}
\date{\today}

\begin{abstract}
    This is a place where I can keep all of my working notes together
    for future use.  The use is primarily for myself, but anyone who
    wants to could use these notes.  I highly doubt that any new
    information is presented within; however, a novel perspective might
    very well be contained herein.
\end{abstract}
\maketitle

%--------1---------2---------3---------4---------5---------6---------7--
\section{Boundary Element Method}
In this section, I want to work out some details for the formulation of
the boundary element method (BEM).  This will be a little sporadic so
bear with me.

\subsection{Green's Equations}
In chapter 1 section 3 of~\cite{ang_beginners_2007}, he blows through a
couple of math steps that I want to work out.  We take \(u\) and \(v\)
to be solutions to the two dimensional Laplace equation
\begin{align*}
    \frac{\partial^2 u}{\partial x^2} + 
    \frac{\partial^2 u}{\partial y^2} &= 0, \\
    \frac{\partial^2 v}{\partial x^2} + 
    \frac{\partial^2 v}{\partial y^2} &= 0. 
\end{align*}
Multiplying the first by \(v\) and the second \(u\) and subtracting we
find
\begin{align*}
    v\frac{\partial^2 u}{\partial x^2} -
    u\frac{\partial^2 v}{\partial x^2} +
    v\frac{\partial^2 u}{\partial y^2} -
    u\frac{\partial^2 v}{\partial y^2} = 0.
\end{align*}
And here is the step he glossed over.  Simply add and subtract
\(\partial_x\,u\partial_xv\) to get
\begin{align*}
    v\frac{\partial^2 u}{\partial x^2} -
    u\frac{\partial^2 v}{\partial x^2} +
    \frac{\partial v}{\partial x}\frac{\partial u}{\partial x} -
    \frac{\partial v}{\partial x}\frac{\partial u}{\partial x} &= 0 \\
    \frac{\partial}{\partial x}\left( 
        v\frac{\partial u}{\partial x} -u\frac{\partial v}{\partial x}
    \right) &= 0
\end{align*}
using the product rule.  A similar process applies to the \(y\)
derivative.  This is just one way to generate Green's functions by the
looks of things.

\subsection{Laplace's Equation}
Okay, so I missed something along the way.  Let me back up and take a
look at deriving the equations in my own notation.  We wish to find a
solution to Laplace's equation in a region \(\Gamma\) bounded by the
surface \(S\).  We assume that either the value or the normal
derivative, but not both, of the field is specified on \(S\) at all
points.  We seek a solution to Laplace's equation that satisfies the
boundary conditions.  Our first task will be to formulate an integral
equation, and then we shall introduce an expansion set to convert to a
matrix formulation.

We begin our investigation in the 2D case.  In two dimensions, the
Laplace operator becomes
\begin{align*}
    \div\grad u(x,y) &= \frac{\partial^2 u}{\partial x^2}
    +\frac{\partial^2 u}{\partial y^2} \\
    \div\grad u(\rho,\theta) &= \frac{1}{\rho}
    \frac{\partial}{\partial\rho} 
    \left( \rho\frac{\partial u}{\partial\rho} \right)
    +\frac{1}{\rho^2}\frac{\partial^2 u}{\partial\theta^2}.
\end{align*}
The next part loosely follows the Wikipedia page on Green's identities,
but I know I've seen it in common texts in the literature.  I just don't
have any handy at the moment.  Now we take the divergence theorem
\begin{equation*}
    \int_\Gamma \div\svec{F}\,dV = \oint_S \unitvec{n}\cdot\svec{F}\,dS
\end{equation*}
with the vector field \(\svec{F} = u\div v -v\div u\) to find
\begin{equation*}
    \int_\Gamma u\div\grad v -v\div\grad u \, dV =
    \oint_S
    u\frac{\partial v}{\partial n} -v\frac{\partial u}{\partial n}
    \,dS.
\end{equation*}
This becomes the starting point for the three dimensional case.  In the
case of two dimensions, \(u\) and \(v\) have no functional dependence on
\(z\) and the Laplace operator is as defined above.  We may remove the
integration over \(z\) by simply inserting the Dirac delta function
\(\delta(z)\) and integrating.  Thus the equation of interest in two
dimensions is 
\begin{equation*}
    \int_\Gamma u\div\grad v -v\div\grad u \, dA =
    \oint_S
    u\frac{\partial v}{\partial n} -v\frac{\partial u}{\partial n}
    \,dS.
\end{equation*}
Boy that was pointless.  The equation looks the same.  I sure hope the
derivation of the Green's function is more interesting.

Next, we introduce the function \(g(\svec{x},\svec{x}')\) that is a
solution to the Poisson equation
\begin{equation*}
    \div\grad g(\svec{x},\svec{x}') = -\delta(\svec{x}-\svec{x}').
\end{equation*}
I know that Rother had a good description of this part for the Helmholtz
equation~\cite{rother_electromagnetic_2009}.  Plugging this into Green's
second identity above and taking \(u\) to satisfy Laplace's equation we
find
\begin{equation*}
    u(\svec{x}) = \oint_S \left[
        g(\svec{r},\svec{x})\grad[\svec{r}] u(\svec{r})
        -u(\svec{r})\grad[\svec{r}] g(\svec{r},\svec{x})
    \right] \cdot \unitvec{n} \, dS(\svec{r})
\end{equation*}
I should really use some equation labels.  In two dimensions, this will
simply reduce to the integral around the contour bounding the region
\(\Gamma\).  Note, that this is for a point \(\svec{x}\) that lies with
in the region \(\Gamma\).  The function \(u\) is not defined external to
this region.  But what happens, when \(\svec{x}\) lies on the surface
\(S\)?  Rodriguez and company state that there is a \(1/2\) but they
don't derive it~\cite{rodriguez_fluctuating-surface-current_2013}, Ang
derives it in 2D in what I can see~\cite{ang_beginners_2007}.  I guess I
should sketch out the proof.

Here we go!  Due to the singularity when \(\svec{x}\) lies on the
surface \(S\), we must interpret the surface integral as a principle
value~\cite{ang_beginners_2007}.  If we surround the point \(\svec{x}\)
by a circle of radius \(\eps\) and exclude this region from surface
integral, we know would get \(0\).  By letting \(\eps\) tend toward
zero, we would get the original form.  So, we need to evaluate an
integral of the form
\begin{equation*}
    \lim_{\eps\to0}
\end{equation*}
I need to look at some more formalisms.  In principle, if we exclude a
region of "radius" \(\eps\) from the surface integral we could then let
\(\eps\) tend toward \(0\).  This would then see a locally flat region
of \(S\) was smooth.  I'm not completely sure how to get that region to
essentially have a factor of \(-1/2\).  I should look into the
references in~\cite{rodriguez_fluctuating-surface-current_2013}.

Now, what is the Green's function in two dimensions?  First, we Fourier
transform the differential equation to the frequency domain
\begin{align*}
    \int_{\mathbb{R}_n} \div\grad g(\svec{r})
    e^{-i \svec{k}\cdot\svec{r}} d^nr &= -\int_{\mathbb{R}_n}
    \delta(\svec{r}) e^{-i\svec{k}\cdot\svec{r}} d^nr \\
    k^2 \tilde{g}(\svec{r}) &= -1 \\
    g(\svec{r}) &= \frac{-1}{(2\pi)^n} \int_{\mathbb{R}_n}
    \frac{e^{i\svec{k}\cdot\svec{r}}}{k^2} \, d^nk
\end{align*}
Now to derive the Green's function in various dimensions.

First, we start with \(n=3\).  This gives us
\begin{align*}
    g(\svec{r}) &= \frac{-1}{(2\pi)^3} \int_0^{2\pi}d\phi_k
    \int_0^\infty \int_{-1}^1 \frac{e^{ikru}}{k^2} \,du \,k^2 dk \\
    &= \frac{i}{(2\pi)^2r}
    \int_0^\infty \frac{e^{ikr} -e^{-ikr}}{k} \,dk \\
    &= \frac{i}{(2\pi)^2r} \int_{-\infty}^\infty \frac{e^{ikr}}{k} \,dk
\end{align*}
where the last step follow from separating at the subtraction,
substituting \(k' \to -k\), swapping the limits of integration,
substituting \(k \to k'\), and adding the integrals.  To perform the
integral, we project onto the complex plane and use the residue
theorem.  We subtract a small negative imaginary part to the denominator
and allow it to approach zero.  We close the contour in the upper half
plane where \(z\) has a large imaginary part so that the integrand goes
to zero.  Thus we find
\begin{align*}
    2\pi i \lim_{\eps\to0} e^{ikr} \Big\rvert_{k=0} &= \lim_{\eps\to0}
    \oint \frac{e^{izr}}{z-i\eps} \,dz \\
    2\pi i&= \int_{-\infty}^\infty \frac{e^{ikr}}{k} \,dk.
\end{align*}

%--------1---------2---------3---------4---------5---------6---------7--
% And the end matter.
\bibliography{notes}
\end{document}

