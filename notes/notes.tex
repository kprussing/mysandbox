\documentclass[aps,prl,reprint]{revtex4-1}

% You'll need my preamble from my dot files.
\input{my_preamble}

\begin{document}
%--------1---------2---------3---------4---------5---------6---------7--
% Header matter
\title{My Working Notes}
\author{Keith Prussing}
\email{kprussing3@gatech.edu}
\affiliation{
    School of Physics \\
    Georgia Institute of Technology, Atlanta, GA, 30332
}
\date{\today}

\begin{abstract}
    This is a place where I can keep all of my working notes together
    for future use.  The use is primarily for myself, but anyone who
    wants to could use these notes.  I highly doubt that any new
    information is presented within; however, a novel perspective might
    very well be contained herein.
\end{abstract}
\maketitle

%--------1---------2---------3---------4---------5---------6---------7--
\section{Boundary Element Method}
In this section, I want to work out some details for the formulation of
the boundary element method (BEM).  This will be a little sporadic so
bear with me.

\subsection{Green's Equations}
In chapter 1 section 3 of~\cite{ang_beginners_2007}, he blows through a
couple of math steps that I want to work out.  We take \(u\) and \(v\)
to be solutions to the two dimensional Laplace equation
\begin{align*}
    \frac{\partial^2 u}{\partial x^2} + 
    \frac{\partial^2 u}{\partial y^2} &= 0, \\
    \frac{\partial^2 v}{\partial x^2} + 
    \frac{\partial^2 v}{\partial y^2} &= 0. 
\end{align*}
Multiplying the first by \(v\) and the second \(u\) and subtracting we
find
\begin{align*}
    v\frac{\partial^2 u}{\partial x^2} -
    u\frac{\partial^2 v}{\partial x^2} +
    v\frac{\partial^2 u}{\partial y^2} -
    u\frac{\partial^2 v}{\partial y^2} = 0.
\end{align*}
And here is the step he glossed over.  Simply add and subtract
\(\partial_x\,u\partial_xv\) to get
\begin{align*}
    v\frac{\partial^2 u}{\partial x^2} -
    u\frac{\partial^2 v}{\partial x^2} +
    \frac{\partial v}{\partial x}\frac{\partial u}{\partial x} -
    \frac{\partial v}{\partial x}\frac{\partial u}{\partial x} &= 0 \\
    \frac{\partial}{\partial x}\left( 
        v\frac{\partial u}{\partial x} -u\frac{\partial v}{\partial x}
    \right) &= 0
\end{align*}
using the product rule.  A similar process applies to the \(y\)
derivative.  This is just one way to generate Green's functions by the
looks of things.

%--------1---------2---------3---------4---------5---------6---------7--
% And the end matter.
\bibliography{notes}
\end{document}

