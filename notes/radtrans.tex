\section{Radiative Transport}
In this section, we will discuss some musings about the general
properties of the radiative transport between grey bodies.  We assume
that all reflections are diffuse.  This implies that for any two bodies
at finite temperatures, the energy exchange between them is 
\begin{equation}
    H_{1,2} = \sigma A_1\mathcal{F}_{1,2} \left(T_1^4 -T_2^4\right).
\end{equation}
In the above, \(A_1\mathcal{F}_{1,2}\) is the total exchange factor 
described by Hottel and Sarofim [I need to figure out which book that
was].  I should really go take a look at those notes I made about all of
this a few years back.  They are on my desktop at work and the text
files are on Ursa.

We know the total exchange factors for a system are dependent on the
geometric view factors \(F_{1,2}\) and the emissivity of the objects.
Once those are know, the total exchange factors are found by solving the
\(N\) systems of equations where \(N\) is the number of ``nodes'' in the
system.  I say ``nodes'' because Ken and I had a huge argument about
using the word ``nodes'' unqualified.  In the current context, I am
referring to a ``thermal node'' concept; however, that is not very
precise. 

Because the total exchange factor \(A_i\mathcal{F}_{i,j}\) capture all
of the multiple reflections, this will give the proper black-body
exchange.  I plotted the exchange normalized by the geometric view
factor \(F_{1,2}\) and saw that the oblate spheroid still displayed an
increase of \(\sim\)80\% while the prolate decrease \(\sim\)40\%.  These
numbers are just off the top of my head and are not precise.  If we
consider the fact that the oblate spheroid has a much larger geometric
view factor, we expect this.  But, because the local normal of an oblate
spheroid is closer to that of a plane, more of the energy can
participate in multiple reflections compared to a sphere or a prolate
spheroid.  This also explains why the sphere-sphere plot was still very
high as opposed to a little high.  Of course, it could simply be that
the near-field effects are huge like everyone says.  This means we do
need to get the total exchange factors and not just the geometric.  This
sounds like a job for vector sweep, but it probably doesn't want to work
with the small length scales (\textsc{F10.4} list directed
\textsc{fortran77} or something like that).  If we can establish that
the exchange factors are scale invariant, we may be able to work with
this.

We may just have to rewrite a program to compute these things.  There
are ``recent'' programs developed to do
this~\cite{walton_calculation_2002}, but I haven't gotten around to
grabbing the source code yet.  A brute force method would be to use a
Monte Carlo approach similar to Whitcher's solid angle
approach~\cite{whitcher_monte_2012}, but we know that is an \(O(N^3)\)
approach~\cite{walton_calculation_2002}.  We could use catalogs for the
well documented geometric view factors~\cite{howell_catalog_2010}, but
that will give us limited precision.  Chung and Naraghi have presented
method on computing the geometric view factor for axially symmetric
cases~\cite{chung_simpler_1982, naraghi_radiation_1982,
chung_formulation_1984}, but that is still the geometric factor.  In our
case, we are only considering two surfaces, so that might not be a
terrible thing once we can establish the emissivity of gold.  I'm still
not sure that we are getting the correct geometric view factor out of
the quick and dirty script.  It only matches the catalog to 2
significant figures~\cite{howell_catalog_2010}.  This leads us back to
scale invariance and vector sweep.

I also want to find the reference Ken was using.  I'm not sure if
Spencer was also using that reference, but I'll never know.  Spencer was
terrible about writing things down.  I know he use \textsc{pbrt}, but
that's all I know.  According to Wikipedia, the geometric view factor
between differential areas is
\begin{equation}
    F_{1\to2} = \frac{\cos\theta_1\cos\theta_2}{\pi S^2} dA_2.
\end{equation}
This is not very symmetric, but it probably is due to the reciprocity
relation \(A_1F_{1\to2} = A_2F_{2\to1}\).  The angles are those between
the normal vector for the surface patch and the vector connecting the
two center points \(\svec{S}\) and \(S\) is the distance between the two
\(S = \lVert\svec{S}\rVert\).  This is all well and good and I remember
this.  I need to dig around in how scaling the differential area
\(dA_2\) works.  Obviously, the separation is easily scaled \(S' = R S\)
where \(R\) is simply a desired ``scaling'' factor.  I use \(R\) because
I know I'll want to use the radius.

